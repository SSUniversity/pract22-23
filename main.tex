\documentclass[master, och, pract]{SCWorks}
% Тип обучения (одно из значений):
%    bachelor   - бакалавриат (по умолчанию)
%    spec       - специальность
%    master     - магистратура
% Форма обучения (одно из значений):
%    och        - очное (по умолчанию)
%    zaoch      - заочное
% Тип работы (одно из значений):
%    coursework - курсовая работа (по умолчанию)
%    referat    - реферат
%    otchet     - универсальный отчет
%    nirjournal - журнал НИР
%    diploma    - дипломная работа
%    pract      - отчет о научно-исследовательской работе
%    autoref    - автореферат выпускной работы
%    assignment - задание на выпускную квалификационную работу
%    review     - отзыв руководителя
%    critique   - рецензия на выпускную работу
% Включение шрифта
%    times      - включение шрифта Times New Roman (если установлен)
%                 по умолчанию выключен
\usepackage{preamble}

\begin{document}

% Кафедра (в родительном падеже)
\chair{Информатики и программирования}

% Тема работы
\title{Разработка платформы единого резюме}

% Курс
\course{1}

% Группа
\group{173}

% Факультет (в родительном падеже) (по умолчанию "факультета КНиИТ")
% \department{факультета КНиИТ}

% Специальность/направление код - наименование
% \napravlenie{02.03.02 "--- Фундаментальная информатика и информационные технологии}
\napravlenie{02.04.03 "--- Математическое обеспечение и администрирование информационных систем}
% \napravlenie{09.03.01 "--- Информатика и вычислительная техника}
% \napravlenie{09.03.04 "--- Программная инженерия}
% \napravlenie{10.05.01 "--- Компьютерная безопасность}

% Для студентки. Для работы студента следующая команда не нужна.
% \studenttitle{Студентки}

% Фамилия, имя, отчество в родительном падеже
\author{Кулакова Максима Сергеевича}

% Руководитель НИР
\nirtitle{к.\,э.\,н., доцент} % степень, звание
\nirname{Л.\,В.\,Кабанова}

% Заведующий кафедрой
\chtitle{к.\,ф.\-м\,н., профессор} % степень, звание
\chname{Д.\,К.\,Андрейченко}

% Научный руководитель (для реферата преподаватель проверяющий работу)
\satitle{к.\,э.\,н., доцент} %должность, степень, звание
\saname{Л.\,В.\,Кабанова}

% Руководитель практики от организации (только для практики, для остальных типов
% работ не используется)
\patitle{к.\,ф.-м.\,н., доцент}
\paname{А.\,С.\,Иванова}

% Семестр (только для практики, для остальных типов работ не используется)
\term{1}

% Наименование практики (только для практики, для остальных типов работ не
% используется)
\practtype{учебная}

% Продолжительность практики (количество недель) (только для практики, для
% остальных типов работ не используется)
\duration{19}

% Даты начала и окончания практики (только для практики, для остальных типов
% работ не используется)
\practStart{01.09.2022}
\practFinish{15.01.2023}

% Год выполнения отчета
\date{2023}

\maketitle

% Включение нумерации рисунков, формул и таблиц по разделам (по умолчанию -
% нумерация сквозная) (допускается оба вида нумерации)
\secNumbering

\tableofcontents

% Раздел "Обозначения и сокращения". Может отсутствовать в работе
% \abbreviations
% \begin{description}
%     \item ... "--- ...
%     \item ... "--- ...
% \end{description}

% Раздел "Определения". Может отсутствовать в работе
% \definitions

% Раздел "Определения, обозначения и сокращения". Может отсутствовать в работе.
% Если присутствует, то заменяет собой разделы "Обозначения и сокращения" и
% "Определения"
% \defabbr

% Ссылка на источник в тексте
% \cite{}

\intro
Вопрос поиска работы всегда находился перед лицом человека, ведь работа должна приносить 
не только деньги, но и удовлетворение физических и психологических потребностей человека.
Наиболее актуальной проблемой со стороны соискателя является то, где искать необходимого 
ему работодателя, а также с какой стороны преподнести свои навыки и умения, чтобы в 
ближайшие дни занимать рабочее место своей мечты. 

В настоящее время для поиска работы в интернете существует множество сервисов, 
при помощи которых работодатели могут как выкладывать свои вакансии, так и рассматривать 
входящие предложения от соискателей. Некоторые из платформ для поиска работы 
предоставляют возможность создавать резюме напрямую в личном кабинете соискателя.

Из-за того, что существует огромное количество сервисов для поиска работы, кому-то 
из работодателей не всегда будет удобна та или иная платформа, и своё предпочтение 
он отдаст “третьей”. Соискателю становится труднее ориентироваться в сайтах для поиска 
нужной вакансии, особенно если резюме на них висит уже долгое время, а навыки человека 
успели обновиться, что требует уже его дальнейшее обновление. Держать под контролем 
несколько резюме на нескольких сайтах становится всё сложнее,что становится проблемой 
для активного соискателя.

В качестве очевидного решения данной ситуации является разработка платформы единого резюме 
с привязкой и последующим его обновлением на различных сервисах поиска работы (таких 
как GitHub, Хабр Карьера, HH.ru).

Целями научно-исследовательской работы являются следующие пункты:
\begin{enumerate}
    \item Обзор научной литературы (в том числе научно-технической) по теме 
    “Разработка платформы единого резюме”;
    \item Рассмотрение и анализ существующих платформ для создания резюме;
    \item Формулировка собственных методов разработки единой платформы резюме;
    \item Подведение итогов проведенной научно-исследовательской работы.
\end{enumerate}


% После введения — серии \section, \subsection и т.д.
\section{Анализ научной литературы}
Рассматриваемая литература будет затрагивать тему аспектов составления резюме, принципы 
их составления и критерии, по которым работодателю с наибольшей вероятностью понравится 
грамотно составленное резюме. После проведения анализа данной темы нам предоставится 
возможность выделить основные пункты, которые будут учитываться в разработанной нами 
единой платформе резюме. 

Для начала стоит рассмотреть научные статьи, связанные с доказательством важности 
правильного составления резюме в настоящее время, и какие изменения  оно претерпевает. 
В статье К.В. Косолаповой “Типологические особенности современного резюме на английском 
языке” автор выделяет основные пункты в резюме, которые было принято считать достаточными:
\begin{enumerate}
    \item Полные ФИО;
    \item Возраст;
    \item Место проживания на текущий момент;
    \item Место учёбы, уровень образования;
    \item Список умений;
    \item Опыт работы (при его наличии);
    \item Контактные данные.
\end{enumerate}

Автор также подчёркивает, что резюме является “визитной карточкой” для работодателя, 
и в зависимости от того, как она составлена, будет зависеть решение работодателя 
о приглашении соискателя на собеседование. В статье раскрывается современное 
понятие “резюме”, а также подмечается, что постепенное развитие и становление 
резюме определило его типологическое разнообразие. Так, традиционными типами 
резюме принято считать хронологическое (Chronological CV), функциональное 
(Functional CV) и комбинированное (Combination CV). После обзора на каждый из 
типов резюме автор рекомендует придерживаться следующим пунктам при 
составлении документа:
\begin{enumerate}
    \item Личные сведения в начале резюме (имя, адрес, контактные телефоны, 
    адрес электронной почты); 
    \item Краткое описание себя от первого или третьего лица в виде небольшого параграфа;
    \item Список ключевых навыков, умений соискателя, а также занимаемых ранее должностей;
    \item Опыт работы, профессиональные успехи, поставленные цели;
    \item Образование, курсы, стажировки, академические степени, 
    квалификации, членство в профессиональных организациях; 
    \item Дата рождения, пол, наличие водительского удостоверения; 
    \item Хобби и интересы.
\end{enumerate}

Также, опираясь на зарубежные исследования, автор упоминает ряд рекомендаций по 
составлению резюме таким образом, чтобы работодатель с наибольшей вероятностью 
выбрал именно Вас на желаемую Вами вакансию.

Что касается сохранения единого резюме для вакансий, однообразие может 
конфликтовать с желанием соискателя попробовать себя в альтернативной сфере. 
Так автор Е.А.Шинкаренко в статье “Резюме как элемент практик поиска работы молодежью” 
в своей статье описывает эволюцию популярных вакансий, по которым чаще откликаются 
молодые люди возрастом 18-25 лет. Несмотря на смену интересов современной молодежи, 
в последнее время работодатели ставят более строгие критерии на свои вакансии, 
вследствие чего будущий соискатель должен будет обладать (в большинстве случаев) 
уровнем образования не ниже незаконченного высшего. В своём выводе автор статьи 
подчеркивает, что стремление молодых соискателей в профессиональные сферы при 
составлении нескольких резюме можно рассматривать как стратегию адаптации к разнообразию 
вакансий на рынке труда, применение образовательных компетенций к разным типам 
трудовой деятельности, разному содержанию труда, но то же время это делает рынок 
труда и профессий очень дифференцированным, что порождает стремление попробовать 
себя в разных его сегментах. Так для выпускника высшего образовательного 
учреждения одно из резюме может быть попыткой реализовать диплом, но при этом 
другое резюме может быть составлено абсолютно по другой траектории развития 
(например, творческая специальность, связанная с хобби, вместо технической). 
Это можно рассматривать как желание занять максимально выгодные позиции, 
так и с точки зрения снижения рисков в условиях трансформации рынка труда 
и появления новых профессиональных сфер.


\section{Анализ конкурентных платформ}
Существуют решения, предлагающие создание резюме на своей платформе с
различными плюсами и минусами. Рассмотрим самые популярные и востребованные решения.

\subsection{HH.ru}
HeadHunter является одним из крупнейших сервисов по поиску работы и сотрудников в России 
и по всему миру. Каждый месяц на сайте обрабатывается свыше сотни тысяч вакансий, 
и ещё большее количество людей имеют возможность найти работу мечты. 
Со стороны соискателей алгоритм отклика на вакансию выглядит следующим образом:

\begin{enumerate}
    \item Зайти в свой аккаунт на hh.ru;
    \item Найти кнопку “Создать резюме”;
    \item Заполнить пункт с контактными данными. Структура пункта представлена 
    на рисунке~\ref{fig:1}:
        \begin{figure}[!ht]
            \centering
            \includegraphics[width=12cm]{images/image14.png}
            \caption{\label{fig:1}%
                Структура пункта контактных данных}
        \end{figure}

    \item Заполнить пункты основной информации. Структура пункта представлена 
    на рисунке~\ref{fig:2}:
        \begin{figure}[!ht]
            \centering
            \includegraphics[width=12cm]{images/image12.png}
            \caption{\label{fig:2}%
                Структура пункта основной информации}
        \end{figure}

    \item Указать желаемую специальность и заработную плату. Структура пункта 
    представлена на рисунке~\ref{fig:3}:
        \begin{figure}[!ht]
            \centering
            \includegraphics[width=12cm]{images/image3.png}
            \caption{\label{fig:3}%
                Структура пункта специальности}
        \end{figure}

    \item Указать опыт работы (при его наличии), а также навыки, которые 
    предлагаются пользователю в качестве отдельных ключевых слов;
    \item Указать уровень образования, место его получения и года выпуска 
    (либо “по настоящее время” для школьников или студентов);
    \item Указать владение языками и его уровень (для иностранных). Структура пункта 
    представлена на рисунке~\ref{fig:4}:
        \begin{figure}[!ht]
            \centering
            \includegraphics[width=12cm]{images/image1.png}
            \caption{\label{fig:4}%
                Структура пункта владения языками}
        \end{figure}

    \item Пункт “другой важной информации” содержит в себе сведения о готовности к переезду, 
    желаемой занятости, графика работы, наличии автомобиля и водительских прав, а также 
    категорий в них. Для иностранных граждан присутствует пункт “разрешения на работу”.
\end{enumerate}

После публикации резюме его может быть не видно большинству работодателей, 
если некоторые из пунктов являются незаполненными. Сам алгоритм составления резюме 
не является сложным, а возможность откликнуться на вакансию часто подразумевает 
прикрепление сопроводительного письма помимо самой “визитки” соискателя.
Дополнительно сервис hh.ru предлагает услуги экспертов для составления грамотного 
резюме за небольшую плату (от 3 до 8 тысяч рублей в зависимости от разновидности услуги).

\subsection{Habr Карьера}
Являясь одним из популярных в России коллективным IT-блогом, Хабр смог развить не только 
форум для программистов, но и отдельные сервисы, которые связаны с помощью начинающих 
и опытных разработчиков, тестировщиков, дизайнеров и прочих информационных вакансий. 
Одним из подобных сервисов для поддержки начинающих и опытных IT-специалистов является 
Habr Карьера.

В отличие от hh.ru, Habr Карьера публикует вакансии исключительно связанные с IT сферой. 
На сервисе предлагаются вакансии как небольших компаний, так и компаний-гигантов 
(например, Яндекс, Авито, Mail.ru). Для своих коллег и знакомых на Habr.Карьера 
пользователю предоставляется возможность оставить профессиональную рекомендацию.

Составление резюме на сайте начинает свой путь с процесса авторизации на сервисе. 
Это можно сделать как при помощи стандартной регистрации с подтверждением почты, 
так и через сервисы, доступные в России (Вконтакте, Google Account).

Составить своё резюме Habr.Карьера предлагает сразу же после авторизации, 
причём существует возможность импортировать резюме с сервиса hh.ru. Данное окно 
представлено на рисунке~\ref{fig:5}:
\begin{figure}[!ht]
    \centering
    \includegraphics[width=12cm]{images/image9.png}
    \caption{\label{fig:5}%
        Предложение импорта резюме с hh.ru}
\end{figure}


Для ручного ввода или создания первого резюме пользователю потребуется 3 шага:
\begin{enumerate}
    \item На первом указывается фамилия и имя, пол и дата рождения, а также основная цель 
    регистрации на Хабр Карьера. Структура пункта представлена на рисунке~\ref{fig:6}:
    \begin{figure}[!ht]
        \centering
        \includegraphics[width=12cm]{images/image15.png}
        \caption{\label{fig:6}%
            Первый шаг регистрации}
    \end{figure}

    В той же вкладке (если выбрана роль соискателя) необходимо выбрать основную 
    специализацию и отдельный профиль, а также квалификация (Intern, Junior, Middle, 
    Senior, Lead). В отличие от hh.ru и в связи с ограниченной сферой деятельности, 
    все специализации представлены наглядно и распределены по категориям для удобства выбора.
    Структура пункта представлена на рисунке~\ref{fig:7}:
    \begin{figure}[!ht]
        \centering
        \includegraphics[width=12cm]{images/image6.png}
        \caption{\label{fig:7}%
            Выбор спициализации и квалификации}
    \end{figure}

    Дополнительно указываются профессиональные навыки, которым владеет соискатель. 
    Список таких навыков очень обширен и позволяет гибко найти нужные профессиональные 
    “скиллы”. Часть из них предлагается уже на старте как “Самые популярные”. 
    Структура пункта представлена на рисунке~\ref{fig:8}:
    \begin{figure}[!ht]
        \centering
        \includegraphics[width=12cm]{images/image19.png}
        \caption{\label{fig:8}%
            Указание профессиональных навыков}
    \end{figure}

    \item Вторая вкладка будет содержать в себе контактную информацию, что также важно 
    при составлении резюме, ведь одних данных об имени и фамилии будет недостаточно 
    для связи с соискателем. По содержанию оно аналогично сервису hh.ru, например, 
    указать город проживания и пункт о готовности к переезду или удаленной работе. 
    Однако для связи соискатель может оставить не только номер телефона или 
    ссылку-портфолио, но и свой логин в мессенджерах, например, в Telegram.
    Структура пункта представлена на рисунке~\ref{fig:9}:
    \begin{figure}[!ht]
        \centering
        \includegraphics[width=12cm]{images/image2.png}
        \caption{\label{fig:9}%
            Указание контактных данных}
    \end{figure}

    \item Последняя вкладка содержит в себе пункты, связанные с опытом работы. 
    По статистике, большинство работодателей присматриваются к кандидатам, 
    у которых за спиной есть даже самый незначительный, но указанный в резюме опыт. 
    Однако на Хабр Карьера отсутствует пункт, связанный с “фрилансом”, что мешает 
    заполнению опыта работы в данной сфере, но его можно добавить самостоятельно.
\end{enumerate}

\subsection{Skipp.dev}
Skipp.dev является международным сервисом как для поиска работы, так и для найма 
работников со стороны работодателей. Различие между предыдущими сервисами отличается 
не только в плане дизайна, но и порядка заполнения пунктов. Для того, чтобы начать 
поиск желаемой вакансии, необходимо также пройти авторизацию, и после этого 
поочередно заполнить пункты специализации и навыков:
\begin{enumerate}
    \item Необходимо верифицироваться по номеру мобильного телефона. 
    Этот пункт необходим для того, чтобы проверить подлинность аккаунта, 
    а также для дальнейшей авторизации на сервисе при помощи смс-кода.
    \item На первом шаге соискателю предлагается выбрать свои навыки. Структура 
    пункта представлена на рисунке~\ref{fig:10}:
        \begin{figure}[!ht]
            \centering
            \includegraphics[width=12cm]{images/image10.png}
            \caption{\label{fig:10}%
                Выбор навыков}
        \end{figure}

    \item После выбора навыков необходимо указать уровень навыков, 
    которые были выбраны на предыдущей странице, что является нестандартным 
    для отечественных сервисов. Структура пункта представлена на рисунке~\ref{fig:11}:
    \begin{figure}[!ht]
        \centering
        \includegraphics[width=12cm]{images/image11.png}
        \caption{\label{fig:11}%
            Указание уровня навыков}
    \end{figure}

    \item Заполнение опыта работы не сильно отличается от рассмотренных 
    нами предыдущих платформ, но имеет свои преимущества. Например, 
    для более подробного описания должностных обязанностей на предыдущих 
    местах работы пользователю предлагаются различные виды деятельности, 
    также начиная с самых популярных на сайте. Также к опыту работы есть возможность 
    приложить изображения в качестве портфолио. Структура пункта представлена 
    на рисунке~\ref{fig:12}:
    \begin{figure}[!ht]
        \centering
        \includegraphics[width=12cm]{images/image17.png}
        \caption{\label{fig:12}%
            Указание опыта работы}
    \end{figure}

    \item Последний пункт содержит в себе форму заполнения контактной информации, 
    начиная от ФИО до номера телефона и e-mail. 
\end{enumerate}

Из положительных сторон данного сервиса можно выделить приятный внешний вид 
и неторопливое пошаговое заполнение всех пунктов. Акцент делается на опыте работы, 
чтобы сконцентрировать внимание пользователя на том, как будет лучше преподнести 
себя будущим работодателям. Однако сервис не захватывает резюме с других платформ, 
что делает эту платформу абстрагированной от других себе подобных.

\subsection{Icanchoose}
ICanChoose рекомендует себя в качестве карьерного ресурса нового формата. Из новизны, 
предполагаемо, сервис предлагает помощь в поиске работы, а также предлагает карьерные советы.

Авторизация на сервисе доступна при помощи ВКонтакте, а так же с email.
При создании резюме сразу предлагается импорт из сервиса hh.ru, представленный 
на рисунке~\ref{fig:13}:
\begin{figure}[!ht]
    \centering
    \includegraphics[width=12cm]{images/image13.png}
    \caption{\label{fig:13}%
        Возможность импорта резюме}
\end{figure}

Процесс составления резюме происходит так же, как на вышерассмотренных сервисах, 
но с добавлением пунктов о хобби, достижениях и сопроводительное письмо. 
Также на сервисе существует возможность предварительного просмотра готового резюме, 
которое работает только при полном заполнении основной информации.

\subsection{ГитХаб}
ГитХаб зарекомедовал себя крупнейшим сервисом для хостинга с возможностью их совместной 
разработки, но, благодаря своим инструментам, на платформе возможно создание резюме. 
На аккаунте начинающего разработчика такое решение будет являться хорошим продолжением 
стратегии его развития, так как при рассмотрении профиля в качестве портфолио работодателю 
будут видны все разработанные проекты, на основе чего высока вероятность получить 
приглашение самому разработчику.

Для начала работы с составлением резюме пользователю необходимо создать новый репозиторий 
с названием, которое будет повторять “юзернейм” на GitHub. Сервис подчеркнёт его в качестве 
уникального и захватит его нужным образом.

Вся информация по резюме будет находиться в файле README.md. Другими словами, 
всё написанное и отформатированное будет видно на странице в GitHub и будет служить 
в дальнейшем красочной визиткой для тех, кто будет просматривать профиль разработчика. 

Удобство написания резюме в GitHub подкрепляется отсутствием шаблона и полной свободой мысли, 
однако из-за отсутствия точных критериев содержания информации необходимо для себя составлять 
структуру будущей визитки, чтобы она выглядело не только гармонично, 
но и понятно \cite{lienhart_maydt_2002}.

\subsection{Tilda}
Tilda изначально является конструктором, позволяющиим на внутренних шаблонах создать 
любой сайт, который удовлетворяет задаче, в том числе резюме.

Для последнего на сайте есть готовые шаблоны, которые упрощают работу с сервисом, 
в частности для начинающих пользователей. Визуально это представлено на рисунке~\ref{fig:14}:
\begin{figure}[!ht]
    \centering
    \includegraphics[width=12cm]{images/image7.png}
    \caption{\label{fig:14}%
        Предложение использования шаблона}
\end{figure}

Составленный из блоков шаблон также можно будет использовать в качестве собственного лендинга.

Профиль на Тильде также является портфолио, что позволяет работодателю, желающему создать 
сайт на данном сервисе, найти необходимого ему разработчика при помощи системы 
заказов на Tilda Express.

Резюме на Tilda Express представляет из себя небольшую страницу с размещенными на 
них проектами, разработанными на Tilda, а также окошком с заполнением заявки на обратную 
связь при наличии желания заказать у данного разработчика собственный сайт. Пример резюме 
представлен на рисунке~\ref{fig:15}:
\begin{figure}[!ht]
    \centering
    \includegraphics[width=12cm]{images/image8.png}
    \caption{\label{fig:15}%
        Пример резюме на платформе Tilda}
\end{figure}

\subsection{Simpledoc}
SimpleDoc является сервисом исключительно по составлению резюме. На сайте предлагается 
составление резюме в трёх основных этапах: заполнение формы, выбор дизайна для резюме 
и его скачивание или отправка по e-mail, представленных на рисунке~\ref{fig:16}:
\begin{figure}[!ht]
    \centering
    \includegraphics[width=12cm]{images/image5.png}
    \caption{\label{fig:16}%
        Этапы создания резюме}
\end{figure}

Сам процесс заполнения формы занимает около 10ти минут и содержит в себе стандартные 
для резюме пункты:
\begin{enumerate}
    \item Основная информация;
    \item Личная информация;
    \item Опыт работы;
    \item Образование;
    \item Курсы и тренинги;
    \item Иностранные языки и компьютерные навыки;
    \item Дополнительная информация.
\end{enumerate}

После заполнения пунктов сервис предлагает выбрать один из четырёх шаблонов 
для дальнейшего скачивания, представленных на рисунке~\ref{fig:17}:
\begin{figure}[!ht]
    \centering
    \includegraphics[width=12cm]{images/image4.png}
    \caption{\label{fig:17}%
        Вариант шаблона резюме}
\end{figure}

После составления резюме по шаблону пользователь сможет скачать его и редактировать 
в течение месяца после разовой оплаты на самом сервисе.
Сервис также не позволяет импортировать резюме с других сервисов.

\subsection{Enhancv}
Enhancv также является сервисов для составления резюме, но в отличие от SimpleDoc 
имеет приятный дизайн, представленный на рисунке~\ref{fig:18} уже со стартовой страницы:
\begin{figure}[!ht]
    \centering
    \includegraphics[width=12cm]{images/image18.png}
    \caption{\label{fig:18}%
        Внешний вид сервиса Enhancv}
\end{figure}

При начале работы нам помогает “виртуальный” помощник, показанный на рисунке~\ref{fig:19}, 
с которым мы заполняем пункты по очереди:
\begin{enumerate}
    \item Имя Фамилия
    \item Наличие опыта в создании резюме (для возможности его импортировать)
    \item Готовые шаблоны для будущего резюме.
\end{enumerate}

\begin{figure}[!ht]
    \centering
    \includegraphics[width=12cm]{images/image16.png}
    \caption{\label{fig:19}%
        Помощник в составлении резюме}
\end{figure}

После чего предлагается отредактировать информацию в резюме напрямую в одном из шаблонов, 
показанном на рисунке~\ref{fig:20}:
\begin{figure}[!ht]
    \centering
    \includegraphics[width=12cm]{images/image20.png}
    \caption{\label{fig:20}%
        Внешний вид шаблона резюме}
\end{figure}

На выбор предлагается множество инструментов для комфортной работы с шаблоном: 
от возможности редактирования шрифта до редактируемой инфографики. 
Также является платным сервисом.

На основе просмотренных конкурентных платформ можно вынести их основные недостатки:
\begin{enumerate}
    \item Не все сервисы предлагают импорт резюме из других платформ;
    \item Сервисы, работающие исключительно на составление резюме, берут за свои услуги деньги;
    \item Некоторые сервисы при указании опыта работы требуют название официальной 
    компании или должности, которых не зарегистрированы, 
    и не всегда удается указать нужное название;
    \item В дополнение к пункту 3, не везде возможно указание опыта “свободных заказов” 
    вне сервисов по типу “Фриланс.ру”
    \item Для русскоязычного сегмента сервисов поиска работы основным является именно 
    HeadHunter, импорт резюме с которого предлагается во множестве остальных платформ.
\end{enumerate}


\section{Основные аспекты разработки платформы единого резюме}
С учётом собранных данных при анализе научной литературы и конкурентных сервисов 
для разработки собственной платформы единого резюме необходимо придерживаться следующим пунктам:
\begin{enumerate}
    \item Реализовать сбор информации о текущих резюме с уже существующих сервисов;
    \item Предоставить возможность объединения, замены, дополнения или удаления 
    смежных или отсутствующих пунктов данных;
    \item Создание единого бланка с отредактированными или созданными полями, 
    с возможностью хранения их на личном пространстве;
    \item Обновление резюме на сторонних площадках новыми пунктами информации.
\end{enumerate}

Для составления базового и необходимого на первое время функционала, в качестве развития 
платформы единого резюме, следует реализовать следующее:
\begin{enumerate}
    \item Удобный пользовательский интерфейс, соответствующий современным web-стандартам 
    и позволяющий пользоваться функционалом сервиса на любом устройстве, поддерживающим 
    работу с браузерами;
    \item Авторизацию пользователей и агрегацию данных на сервере проекта;
    \item Возможность создания пользователем персональной страницы на основе шаблонов, 
    для возможности доступа к актуальному резюме при переходе по ссылке сервиса;
    \item Автоматическое обновление данных по прошествию заданного периода времени или 
    при обновлении резюме на сторонних сервисах;
    \item Публичное пространство, которое может быть использовано как место просмотра 
    резюме работодателями для предложений о сотрудничестве;
    \item Страницы помощи по взаимодействию с платформой, в том числе написанию резюме 
    и созданию собственного сайта на основе агрегированных данных.
\end{enumerate}


\newpage
\conclusion
В результате проведения исследовательской работы мы овладели навыками анализа качества 
и эффективности научной литературы в области разработки сервисов с автоматическим 
обновлением данных, достигли навыка анализирования конкурентных платформ для 
создания резюме и сформулировали собственные методы разработки единой платформы резюме, 
тем самым достигнув полного выполнения поставленных нами пунктов.

Анализ конкурентных платформ позволил выявить слабые стороны существующих сервисов, 
исправление которых возможно реализовать в разработке собственной единой платформы 
резюме при условии его дальнейшего масштабирования.
Научно-исследовательская работа помогла приобрести знания для получения дополнительных 
компетенций веб-разработчика. 


% Библиографический список, составленный вручную, без использования BibTeX
%
% \begin{thebibliography}{99}
%   \bibitem{Ione} Источник 1.
%   \bibitem{Itwo} Источник 2
% \end{thebibliography}

% Отобразить все источники. Даже те, на которые нет ссылок.
% \nocite{*}

% Меняем inputencoding на лету, чтобы работать с библиографией в кодировке
% `cp1251', в то время как остальной документ находится в кодировке `utf8'
\inputencoding{cp1251}
\bibliographystyle{gost780uv}
\bibliography{thesis}
\inputencoding{utf8}

% При использовании biblatex вместо bibtex
% \printbibliography

% Окончание основного документа и начало приложений Каждая последующая секция
% документа будет являться приложением
\appendix

\end{document}
